\documentclass[a4paper, 12pt]{article}

\usepackage{indentfirst}
\frenchspacing

\usepackage{cmap}

\usepackage{geometry}
\geometry{left=2cm}
\geometry{right=1.5cm}
\geometry{top=2cm}
\geometry{bottom=2cm}


\begin{document}

\begin{center}

\


{\large
\textbf{Final graduation thesis}

\

``Sales record IT system for enterprise `Sterh' ltd.''
}

\


Far Eastern State Academy for Humanities and Social Studies

Faculty of Mathematics, Information Technologies and Engineering

\

\textbf{Student:} Grigorev Alexey

\textbf{Mentor:} Surmenko Sergey L.

\

\

\

\

{\large\textbf{Summary}}

\end{center}

\

\

The main goal of the project is to automate the sales records information flow in ``Sterh, ltd'', Birobidzan, Russia.

\


Resolved tasks:
\begin{itemize}
\item Analysis of the company's organizational structure.
\item Analysis of its business processes.
\item Analysis of the level of automation. Identification possible subjects of automation.
\item Analysis of existent software and information systems.
\item Creating the conceptual part of the system, use-cases analysis.
\item Designing a database structure.
\item Implementation of an end-user application.
\item Calculation of economic efficiency of the project.
\end{itemize}

\

The work on the project was started on 01.12.2009 and finished on 02.10.2010.


\newpage

Sterh ltd. in Birobidzhan, Russia is a retailing company, and due to rise of many competitors, it was willing to optimize and accelerate its business processes. The initial goal of this thesis was to analyze the existent information flows, find areas to improve, choose one, and create an information system tailored to the needs of this company.

During the research a few discoveries were made, but the most disorganized activity was the commodity circulation flow. Records were kept in an old DOS system which were overfilled with unneeded data. Not only was it very slow, but also counterintuitive for its users and not reliable. What is more, many collateral records were kept in Excel spread sheets which made it impossible to index the data and search there. Also, as a part of this flow another system was used, and the data was transferred manually from one system to another.

Therefore, the main goal of the graduation work is to create an automated information system for sales records flow.

This flow contains many steps, starting from interaction with suppliers and making sure deliveries are made wholly and on time, to receiving orders and delivering them to customers' offices. Additionally, each step is accompanied with supportive documentation, information from some of them eventually has to be filled into the company's accounting system. For the support of the flow, among others, the following reports are needed: what is in the warehouse, what are the current shortages, what is to be delivered as operation reports; what is in the biggest demand, what is not popular, which items are likely to be needed soon as strategic. Additionally, a list of items to order from supplier is also needed.

\

Methods and instruments of accomplishing the goal:

\begin{itemize}
\itemsep0pt
\item analyzing and designing business processes using CASE-tools BPWin, IDEF0 and DFD notations;
\item designing the application model with CASE-tools Rational Roses and the UML notation;
\item designing database schemata using CASE-tools ERWin;
\item developing the end-user application in ``Microsoft Visual Studio 2008'' programing environment, in a high-level programming language C\# plus the Firebird database engine.
\end{itemize}

The thesis consists of three chapters.

The first chapter is analytical, which analyzes organizational and economical structures of Sterh ltd. as well as its level of automation of information flows. It discovers objects and subjects of possible automation, and as the result of this chapters, the object is identified (the goods flow), and a possible solution is suggested.

The second chapter of the thesis contains information about collecting use-cases, building a conceptual model, creating high-level design diagrams. In this chapter the database structure is describes as well as the main ideas of the application. Additionally, legal and security problems are considered.

In the third chapter economic efficiency of the system is calculated. The main idea of the calculation is time saved by users as a result of introducing the system. Productivity of the users was increased by 25\% compared to the old systems, and the payback period of the product is estimated 15 months.

The main practical result of the final graduate thesis is the automated information system for sales records. This system was successfully adopted by Sterh in the autumn of 2010.

\end{document}
